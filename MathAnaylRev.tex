\documentclass[UTF8]{ctexart}
    \usepackage{amsmath}
    \usepackage{hyperref}
    \usepackage{indentfirst}
    \usepackage{cite}
    \usepackage{multicol}
    \usepackage{geometry}
\title{数学分析复习}
\begin{document}
    \pagestyle{plain}
    \maketitle
    \tableofcontents
\part{数项级数}
\section{定义}
\section{正项级数判别法}
    \subsection{Cauchy判别法}
    \subsection{等比级数}
        \subsubsection{比值判别法}
        \subsubsection{根式判别法}
    \subsection{P级数}
        \subsubsection{Raabe判别法}
        \subsubsection{Bertrand判别法}
\section{一般级数判别法}
    \subsection{莱布尼茨判别法}
    \subsection{狄利克雷判别法}
    \subsection{阿贝尔判别法}
\section{更序问题}
\part{函数列与函数项级数}
    \section{函数列与函数项级数的收敛性}
        \subsection{逐点收敛性}
        \subsection{一致收敛性}
    \section{一致收敛性判别法}
        \subsection{Cauchy判别法}
        \subsection{函数项级数一致收敛判别法}
            \subsubsection{魏尔斯特拉斯判别法}
            \subsubsection{狄利克雷判别法}
            \subsubsection{阿贝尔判别法}
    \section{一致收敛函数列与一致收敛函数项级数的分析性质}
\end{document}